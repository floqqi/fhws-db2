
Sequelize ist ein objektrelationaler Mapper, der im Rahmen einer Masterarbeit von Sascha Depold entwickelt wurde. Als Laufzeitumgebung wurde Node.js gewählt, in der die aus dem Browser bekannte Programmiersprache JavaScript interpretiert wird. Durch asynchrone Patterns in der Sprache, z.B. Callbacks oder Promises, ist sie sehr gut für die Abarbeitung paralleler Prozesse geeignet.

\subsection{Grundlagen}

\paragraph{Relationale Zugriffsschicht} \hspace{0pt} \\
Sequelize bedient sich des Entwurfsmusters des Data Access Object (DAO). Dies beschreibt die Nutzung spezieller Klassen zum Zugriff auf Datenquellen. Die Klassen definieren dabei die zum einen die Schnittstellen zwischen den Daten und der Geschäftslogik. So kann sichergestellt werden, dass die Schnittstellen selbst beim Austausch des Datenbanksystems (also der Datenquelle) existent und konsistent bleiben (\cite[S. 8]{Piep1}). Aufgabe des DAO ist es unter anderem, die Datenquelle (Datenbank) abzufragen, die Ergebnisse entgegenzunehmen und sie in ein entsprechendes Datenobjekt (Data Transfer Object, DTO) zu konvertieren.

\hspace{1pt}
\begin{lstlisting}[caption=Relationale Zugriffsschicht]
class StudentDAO
  def self.all
    query = "SELECT * FROM students"
    return SQLConnector.execurte(query)
  end
  
  def self.update(attributes, id)
    query = "UPDATE students (" + attributes.keys.join(",") + ") VALUES ("
    query += attributes.values.join(",") + ") WHERE id=" + id
    return SQLConnector.execute(query)
  end
end

class StudentController
  def index
    @students = StudentDAO.all
  end
end
\end{lstlisting}


\paragraph{Objektrelationale Zugriffsschicht} \hspace{0pt} \\
Diese Schicht ermöglicht das Arbeiten mit den nun nicht mehr repräsentativen Geschäftsobjekten. Dadurch wird die Manipulation dieser Objekte ermöglicht. Außerdem können nur weitere Features wie z.B. Assoziationen oder Berechtigungen implementiert werden. Auch das Löschen von Datensätzen ist nun möglich.

\hspace{1pt}
\begin{lstlisting}[caption=Objektrelationale Zugriffsschicht]
class StudentDAO
  def self.all
    results = SQLConnector.execute("SELECT * FROM students")
    users = results.map{ |hash| return new UserDO(hash) }
    return users;
  end
  
  def self.find(id)
    result = SQLConnector.execute("SELECT * FROM students WHERE id=" + id)
    return new UserDTO(result)
  end
end

class StudentDTO
  def update(attributes)
    query = "UPDATE students (" + attributes.keys.join(",") + ") VALUES ("
    query += attributes.values.join(",") + ") WHERE id=" + self.id
    return SQLConnector.execute(query)
  end
end

class StudentController
  def index
    @students = StudentDAO.all
  end
end
\end{lstlisting}

\noindent Um doppelten Code zu vermeiden, sollte die Architektur jedoch leicht abgeändert werden. So wird die DAO- bzw. Datenobjekt-Logik in eine Oberklasse ausgelagert und entsprechende Vererbung eingeführt (\cite[S. 11]{Depo1}).


\subsection{Architektur}

\paragraph{Sequelize} \hspace{0pt} \\
Die Klasse Sequelize ist der zentrale Einstiegspunkt des Mappers. Sie ermöglicht das Definieren von DAO-Factories und Konfiguration.

\paragraph{DAOFactory (Model)} \hspace{0pt} \\
Klasse, die die Struktur einer Datenbanktabelle widerspiegelt. Dafür müssen Instanzen den jeweiligen Tabellennamen und die Attribute enthalten. Die Klasse stellt Methoden bereit, um die Tabelle auszulesen oder eine DAO-Instanz zu erstellen.

\paragraph{DAO (Instance)} \hspace{0pt} \\
Instanzen der DAO-Klasse spiegeln einzelne Datenbankeinträge wieder. Sie werden u.A. durch den Aufruf der build()-oder create()-Method der DAOFactory erzeugt.


\subsection{Benutzung}

\paragraph{Klassendefinition} \hspace{0pt} \\
Sequelize verfolgt den Modell- und den Schemagetriebenen Ansatz zur Definition von Modellen. Dies hat zur Folge, dass alle Models vorher definiert werden müssen.

\begin{lstlisting}[caption=Klassendefinition (Sequelize)]
var Student = sequelize.define('student', {
  firstname: Sequelize.STRING(40)
  lastname: Sequelize.STRING(40)
});	
\end{lstlisting}

\noindent Die Spalten `id`, `created\_at` und `updated\_at` müssen nicht definiert werden, diese werden von Sequelize automatisch erzeugt.


\paragraph{Datensatz abfragen} \hspace{0pt} \\
Sequelize stellt in der Klasse DAOFactory mehrere Methoden bereit, mit denen sich Daten aus Tabellen auslesen lassen, z.B. `find()`, `findAll()` und `count()`. Über Optionen wie z.B. `where`, `limit` oder `offset` können die Suchen jeweils angepasst werden.

\begin{lstlisting}[caption=Datensatz abfragen (Sequelize)]
Student.find({
  where: {
    id: {
      $between: [12000, 14000],
      $notIn: [12900, 13000]
    },
    name: 'Meier'
  }
}).then(function(student) {
  console.log('Student: %j', student);
});	
\end{lstlisting}


\paragraph{Datensatz anlegen} \hspace{0pt} \\
Die Methoden `DAOFactory.build()` und `DAOFactory.create()` ermöglichen es, sowohl temporäre als auch persistente Datenbankeinträge zu erstellen. Mit `DAO.save()` wird der temporäre, ungespeicherte Eintrag in der Datenbank gespeichert.

\begin{lstlisting}[caption=Datensatz anlegen (Sequelize)]
// ungespeichertes Model erstellen
var student = Student.build({
  firstname: 'Max',
  lastname: 'Mustermann'
});

// Model speichern
student.save().then(function() {
  // Und alle so: Yeah!
});	
\end{lstlisting}


\paragraph{Datensatz modifizieren} \hspace{0pt} \\

Über die Methoden `save()` und `destroy()` der DAO-Klasse ist es jederzeit möglich, Objekte zu verändern oder zu löschen.

\begin{lstlisting}[caption=Datensatz ändern (Sequelize)]
	student.firstname = 'Moritz';
	student.save();
\end{lstlisting}

\begin{lstlisting}[caption=Datensatz löschen (Sequelize)]
	student.destroy();
\end{lstlisting}



\paragraph{Assoziationen} \hspace{0pt} \\

Nach dem definieren von Models ist es mit Sequelize möglich, Assoziationen zu definieren. Dabei sind 1:1-, 1:m- und n:m-Assoziationen möglich.

\begin{lstlisting}[caption=Datensatz anlegen (Sequelize)]
	var Student = sequelize.define('student', {/* Definition */}),
    	StudyCourse = sequelize.define('studyCourse', {/* Definition */});
	
	StudyCourse.belongsTo(StudyCourse);
\end{lstlisting}

\noindent Diese Definition würde nun zur Tabelle `student` den Foreign-Key `studyCourseId` hinzufügen, um die 1:m-Beziehung abbilden zu können.




