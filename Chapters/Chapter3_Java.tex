Hibernate ist ein ORM-Framework für Java, dessen Hauptaufgabe die objektrelationale Abbildung ist. Dies ermöglicht Klassen, Objekte und Attribute in relationalen Datenbaken zu speichern und auszulesen, dabei gilt:

\begin{itemize}
	\item Jede Klasse entspricht eine Tabelle
	\item Jedes Objekt entspricht einem Datensatz
	\item Jedes Attribut entspricht einer Spalte.
\end{itemize}
(\cite[S. 280]{Piep1})

\subsection{Grundlagen}

Um ein Tabelle aus einer Java Klasse zu erstellen muss das SQL Statement aus einzelnen Attributen des Java Objekts generiert werden. Dabei liegt aber ein Konzeptioneller Wiederspruch zwischen der Objektorientierten Struktur und der flachen Tabellenstruktur vor. Auch Objektrelationaler Bruch genannt. Hibernate fungiert hier als Verbindungsglied zwischen Java und der Relationalen Datenbank. Indem es automatisch SQL Statements aus Java Objekten abstrahiert.


\subsection{Hibernate Konventionen}

Um ein Java-Objekt mittels Hibernate in einer relationalen Datenbank persistent zu verwalten, muss die Klasse des Objektes einige grundlegende Notationen befolgen. Zunächst muss die Klasse mit der Annotation @Entity versehen werden damit sie in Datenbank gemappt werden kann. „Die Annotation @Entity macht aus einer „normalen“ Klasse eine Klasse, deren Instanzen mit einer JPA-Implementierung persistent gemacht werden können.“(\cite[S. 32]{MW1}) Welche Spalte der Primärschlüssel sein soll muss angegeben werden. Dies geschieht mit der Annotation @Id. Die Annotation des Primärschlüssels kann bei der Instanzvariablen oder bei Getter/ Setter-Paar stehen und definiert damit dem Zugriff der JPA-Implementierung auf das Property. Im unteren Beispiel mit Attribut Matrikelnummer geschehen. (\cite[S. 32]{MW1})

Die Attribute sind als privat deklariert, daher werden entsprechend der Java-Konventionen set/get-Methoden für jedes Attribut definiert, die auch von Hibernate so erwartet werden.(\cite[S. 454]{SSH1}) Zu beachten ist das Hibernate einen default Konstruktor erwartet, also einen Parameterfreien Konstruktor, welcher zwar bei keiner Angebe eines Konstruktors automatisch von Java mit initialisiert wird, aber nochmal explizit angegeben werden muss wenn ein weiterer Konstruktor vorhanden ist.

\hspace{0pt}
\begin{lstlisting}[language=Java]
@Entity
public class Student {
	
	@Id
	private int matrikelnummer;
	private String name;
	private String vorname;
	private String email;
	private String studiengang;
	
	public int getMatrikelnummer() {
		return matrikelnummer;
	}

	public void setMatrikelnummer(int matrikelnummer) {
		this.matrikelnummer = matrikelnummer;
	}

	public String getName() {
		return name;
	}

	public void setName(String name) {
		this.name = name;
	}

	// ...
	
	public Student() {};
	
	public Student(int matrikelnummer, String name, String vorname, String email, String studiengang) {
		this.matrikelnummer = matrikelnummer;
		this.name = name;
		this.vorname = vorname;
		this.email = email;
		this.studiengang = studiengang;
	}
}

\end{lstlisting}

\subsection{Data Access Objekt Pattern}

Über das Entwurfsmuster des DAO trennen wir die Programmlogik  von den technischen Details der Erstellung eines Tabels. Zudem verwenden wir das DAO Pattern zum Speichern, Auslesen etc. unseres Objektes.

\paragraph{Sitzung und Transaktion} \hspace{0pt} \\
Zunächst müssen wir die Aktuelle Session öffnen und die Transaktion beginnen. Dies erreichen wir über die Methoden getCurrentSession und beginTransaction welche ich im unteren Beispiel in der Methode openSessionAndTransaction aufrufe. Als Rückgebewert bekommen wir die aktuelle Sitzung. Beim Schließen der Sitzung ist darauf zu achten das Transaktion zunächst Kommittet wird bevor die Sitzung beendet wird (siehe closeSessionAndTransaction).

\paragraph{Zugriff über die DAO} \hspace{0pt} \\
Über das DAO können wir auf verschiedene weisen auf Daten unseres Objektes zugreifen. So können wir nicht nur Objekte speichern oder löschen sondern auch Datensätze ändern oder nach ihnen suchen.

\hspace{0pt}
\begin{lstlisting}[language=Java]
public class StudentDAO {
	
	private Session session;
	private Transaction transaction;
	
	public StudentDAO() {} 

	public Session openSessionAndTransaction() {
		session = HibernateUtil.getSessionFactory().getCurrentSession();
		transaction = session.beginTransaction();
		return session;
	}
	
	public void closeSessionAndTransaction() {
		transaction.commit();
		session.close();
	}
	
	public Session getSession(){
		return session;
	}
	
	public void save(Student s) {
		getSession().save(s);
	}
	
	public void update(Student s) {
		getSession().update(s);
	}
	
	public Student find(int matrikelnummer) {
		Student s = (Student) getSession().get(Student.class, 
				matrikelnummer);
		return s;
	}
	
	public void delete(Student s) {
		getSession().delete(s);
	}

}

\end{lstlisting}
