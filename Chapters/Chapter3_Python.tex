SQLAlchemy ist ein Objekt-relationaler Mapper für die Programmiersprache Python. Das Framework wurde im Februar 2006 von Michael Bayer unter der MIT Lizenz – als Open Source – veröffentlicht. Seitdem gewann es an sehr großer Beliebtheit, vor allem in der Open Source Welt, aber auch bei großen und bekannten Firmen (\cite[Organizations]{SQLA1}) wie Dropbox, Hulu oder Uber.

\subsection{Allgemein}

\paragraph{Philosophie} \hspace{0pt} \\
SQL databases behave less like object collections the more size and performance start to matter; object collections behave less like tables and rows the more abstraction starts to matter. SQLAlchemy aims to accommodate both of these principles. (\cite[Philosophy]{SQLA1})

\paragraph{Unterstützte Datenbanken (Dialekte)} \hspace{0pt} \\
\begin{itemize}
	\item MySQL
	\item SQLite
	\item PostgreSQL
	\item Oracle
	\item Microsoft SQL Server
	\item …
\end{itemize}

Es ist außerdem relativ einfach einen weiteren Dialekt in SQLAlchemy zu integrieren. SAP hat dies zum Beispiel für seine HANA-Datenbank realisiert (\cite[SQLAlchemy-Hana]{GitH1}).

\subsection{Komponenten}
SQLAlchemy besteht aus zwei alleinstehenden Komponenten: Dem Core und dem ORM.

\paragraph{Core} \hspace{0pt} \\
Der Core ist ein Abstraktionslayer für verschiedene Datenbank-API-Implementierungen, welcher es erlaubt SQL-Abfragen über eine allgemeine Syntax in Python zu schreiben ("`SQL Expression Language"').

\begin{lstlisting}[language=Python]
select([Student.id, Student.martrikelnr])
.where(Student.name == 'Peter')
.order_by(Student.name)
\end{lstlisting}

\hspace{0pt}

\noindent
Außerdem bietet er die Möglichkeit jeden Datentyp von Python auf einen SQL-Datentypen zu "`mappen"'. Ein Beispiel hierfür wäre der Datentyp \textit{Decimal} in Python; dieser wird auf \textit{NUMERIC} in PostgreSQL und \textit{DECIMAL} auf SQL Server konvertiert. Dies hat den Vorteil, dass man mit dem nativen Datentypen der Programmiersprache Python arbeiten kann und sich nicht explizit darum kümmmern muss, welcher Datentyp in der Datenbank hierfür verwendet wird.

\paragraph{ORM} \hspace{0pt} \\
Der ORM (Objekt-relationale Mapper) ermöglicht das klassische „Mappen“ von Datenbank-Tabellen auf Objekte. Er erlaubt nicht nur das Mapping auf einzelne Tabellen, sondern ermöglicht auch Objektorientierte Paradigmen wie Vererbung. Referenzierungen über Fremdschlüssel sind ebenfalls möglich.

\begin{lstlisting}[language=Python]
class Studiengang(Model):
    __tablename__ = 'studiengaenge'

    id = Column(
        Integer(),
        primary_key=True
    )

    name = Column(
        Unicode(),
        nullable=False,
        unique=True
    )


class Student(Model):
    __tablename__ = 'studenten'

    id = Column(
        Integer(),
        primary_key=True
    )

    name = Column(
        Unicode(255),
        nullable=False
    )

    studiengang_id = Column(
        Integer(),
        ForeignKey('studiengaenge.id'),
        nullable=False
    )
    studiengang = relationship(
        Studiengang,
        lazy='joined',
        backref='studenten'
    )
\end{lstlisting}


Anhand dieser Klassendefinitionen können nun zwei Tabellen angelegt werden. SQLAlchemy kann anhand der o. g. Notationen die nötigen SQL-Statements (CREATE TABLE …) erzeugen.

\hspace{1.0pt}

\noindent
Das Anlegen eines neuen Datensatzes sieht nun folgendermaßen aus:

\begin{lstlisting}[language=Python]
studiengang = Studiengang()
studiengang.name = 'Informatik'

student = Student()
student.name = 'Peter'
student.studiengang = studiengang

session.add(student)
session.commit()
\end{lstlisting}

Die INSERT-Statements werden nun mithilfe der "`Unit of Work"' (\cite[S. 94--95]{Cope1}) in der richtigen Reihenfolge ausgeführt, sowie die Fremdschlüsselreferenz von \textit{student.studiengang\_id} automatisch aufgelöst.

\subsection{Unit of Work}

Die Unit of Work ist ein integraler Bestandteil von SQLAlchemy’s ORM-Erweiterung. Diese folgt dem "`Unit of Work"'-Pattern von Martin Fowler (\cite{Fowl1}), welches bspw. auch von Hibernate (Java’s populärster ORM) genutzt wird.

Änderungen an Objekte werden beim Unit of Work-System nicht direkt in die Datenbank geschrieben, sondern gesammelt in einer Transaktion abgeschickt. Hierfür wird eine topologische Sortierung nach Abhängigkeiten durchgeführt, um z. B. Fremdschüsselreferenzen (siehe o. g. Beispiel) aufzulösen. Diese System hat außerdem den Vorteil, dass redundante Anfragen gebündelt und Deadlocks vermieden werden können.
